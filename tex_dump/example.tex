\documentclass{article}
\usepackage{graphicx}
\title{test}
\author{Zevs Grom}
\date{January 2025}
\begin{document}
\maketitle
\section{Let's differentiate this statement} \textbf{\Large 2 * x + 3 / x\normalsize}

It's not hard to notice, that if we differentiate:\begin{equation}
 2 \cdot x 
 \end{equation}
\begin{equation}
+
 \end{equation}
\begin{equation}
\frac{ 3 }{ x}
 \end{equation}

It's not hard to notice, that if we differentiate:\begin{equation}
 2 \cdot x 
 \end{equation}
\\Now, let's differentiate constant:  \texbf{\large 2 \normalsize}
\\Answer is for the intermediate step:  \texbf{\large 0 \normalsize}
\\Now, let's differentiate:  \texbf{\large x \normalsize}
\\Answer is for the intermediate step:  \texbf{\large 1 \normalsize}

We will get:\begin{equation}
 0 \cdot x 
 \end{equation}
\begin{equation}
+
 \end{equation}
\begin{equation}
 2 \cdot 1 
 \end{equation}

It's not hard to notice, that if we differentiate:\begin{equation}
\frac{ 3 }{ x}
 \end{equation}
\\Now, let's differentiate constant:  \texbf{\large 3 \normalsize}
\\Answer is for the intermediate step:  \texbf{\large 0 \normalsize}
\\Now, let's differentiate:  \texbf{\large x \normalsize}
\\Answer is for the intermediate step:  \texbf{\large 1 \normalsize}

We will get:\begin{equation}
 0 \cdot x 
 \end{equation}
\begin{equation}
-
 \end{equation}
\begin{equation}
 3 \cdot 1 
 \end{equation}
\begin{equation}
\frac{}
 \end{equation}
\begin{equation}
 x \cdot x 
 \end{equation}

We will get:\begin{equation}
 0 \cdot x 
 \end{equation}
\begin{equation}
+
 \end{equation}
\begin{equation}
 2 \cdot 1 
 \end{equation}
\begin{equation}
+
 \end{equation}
\begin{equation}
 0 \cdot x 
 \end{equation}
\begin{equation}
-
 \end{equation}
\begin{equation}
 3 \cdot 1 
 \end{equation}
\begin{equation}
\frac{}
 \end{equation}
\begin{equation}
 x \cdot x 
 \end{equation}

Let's simplyfy this statement:\begin{equation}
 0 \cdot x 
 \end{equation}
Answer is for the intermediate step:  \texbf{\large 0 \normalsize}

Let's simplyfy this statement:\begin{equation}
 2 \cdot 1 
 \end{equation}

Okay, let's find solution:\begin{equation}
 2 \cdot 1 
 \end{equation}
Answer is for the intermediate step:  \texbf{\large 2 \normalsize}

Let's simplyfy this statement:\begin{equation}
 0 + 2 
 \end{equation}

Okay, let's find solution:\begin{equation}
 0 + 2 
 \end{equation}
Answer is for the intermediate step:  \texbf{\large 2 \normalsize}

Let's simplyfy this statement:\begin{equation}
 0 \cdot x 
 \end{equation}
Answer is for the intermediate step:  \texbf{\large 0 \normalsize}

Let's simplyfy this statement:\begin{equation}
 3 \cdot 1 
 \end{equation}

Okay, let's find solution:\begin{equation}
 3 \cdot 1 
 \end{equation}
Answer is for the intermediate step:  \texbf{\large 3 \normalsize}

Let's simplyfy this statement:\begin{equation}
 0 - 3 
 \end{equation}

Okay, let's find solution:\begin{equation}
 0 - 3 
 \end{equation}
Answer is for the intermediate step:  \texbf{\large -3 \normalsize}

Let's simplyfy this statement:\begin{equation}
 x \cdot x 
 \end{equation}

Let's simplyfy this statement:\begin{equation}
\frac{ -3 }}
 \end{equation}
\begin{equation}
 x \cdot x 
 \end{equation}

Let's simplyfy this statement:\begin{equation}
 2 +
 \end{equation}
\begin{equation}
\frac{ -3 }}
 \end{equation}
\begin{equation}
 x \cdot x 
 \end{equation}

\maketitle
\section{So, answer is} \textbf{\Large :\begin{equation}
 2 +
 \end{equation}
\begin{equation}
\frac{ -3 }}
 \end{equation}
\begin{equation}
 x \cdot x 
 \end{equation}
}
\end{document}
